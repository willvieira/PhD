% Literature review for the first meeting

\documentclass[a4paper,12pt,twoside]{article}
\usepackage[T1]{fontenc}
\usepackage[utf8]{inputenc}
\usepackage{color,dcolumn,graphicx,hyperref}
\usepackage{wrapfig}
\usepackage[top=2.5cm, bottom=2.5cm, left=2.5cm, right=2.5cm]{geometry}
\usepackage{setspace}
\onehalfspacing
\usepackage{scrextend}
\addtokomafont{labelinglabel}{\sffamily}

%% Packages for Graphics & Figures
\usepackage{graphicx} %%For loading graphic files
\usepackage{float}

%% Math Packages
\usepackage{amsmath}
\usepackage{amsthm}
\usepackage{amsfonts}

\begin{document}

\title{Integrating forest management and species distribution models: a theoretical approach under climate change}

\author{Willian Vieira}

\maketitle

\section{What is going on?}

\textbf{Climate change, obviously}

Defining climate change in one paragraph, its impacts and why forecasting.

Species range will definitely change. But is it really important? If so what will happen?

Surprisingly (at last for me), some species do not follow climate change or tend to migrate slower compared with its climate range, creating a climate debit. It happens mainly in trees species that have a slow demography and limited dispersion, hence a slower migration.

What then would happen if we have a climate debit? Impact on diversity, conservation, productivity? Explain with more details each item.

Independent of any impact, we will here consider the importance of resilience to mitigate these impacts, as well as the time to return to the equilibrium.

\section{What to do?}

But now we know a bit better the context, what should we do? An alternative way to increase forest resilience is to consider forest management.

Present some motivations, advantages and disadvantages in considering forest management.

Forest management in a theoretical view was actually not very much explored and so I consider it a kind of gap we should better explore.

\section{Theories to be considered}

\subsection*{Range dynamics theory}

What theories can help us to describe species range under climate change?

How to integrate forest management in this theory?

\subsection*{Species Interaction: why is it important?}

Explain the role species interaction can play on its distribution range.

\section{How to do it?}

Here we see a briefly presentation of possible methods will be used in the thesis.

\subsection*{Modeling}

Why and how modeling?

\subsubsection*{Integral Projection Models}

\textit{I should be writing and not playing with \LaTeX}

\subsection*{Bayesian approach}

\textit{I should be writing and not playing with \LaTeX}

\section*{Thesis structure}

The first part of the thesis will be a general introduction in French where  I will probably use a part of this document and present the big picutre of my thesis.

The first chapter will try to answer the question \textit{Can forest management increase forest resilience to climate change?}. The paper will work with an analytical and sensitivity analysis in a metapopulation dynamics model to understand the impact of forest management on increasing forest resilience.

In the second chapter I am going to build a landscape model that will consider both forest management and species interaction.

The third chapter I am going to build another model but in a local scale. \textit{I have to find a good biological reason for that}.

The fourth chapter will then integrate both landscape and local model into one. Here I will also track the uncertainty of the model by bayesian approach.  

\end{document}
