% Literature review for the first meeting

\documentclass[a4paper,12pt,twoside]{article}
\usepackage[T1]{fontenc}
\usepackage[utf8]{inputenc}
\usepackage{color,dcolumn,graphicx,hyperref}
\usepackage{wrapfig}
\usepackage[top=2.5cm, bottom=2.5cm, left=2.5cm, right=2.5cm]{geometry}
\usepackage{setspace}
\onehalfspacing
\usepackage{scrextend}
\addtokomafont{labelinglabel}{\sffamily}
\usepackage[round]{natbib} %citation style
\usepackage{tcolorbox}
\usepackage{filecontents}
\usepackage{csquotes}
\usepackage{epigraph}
\usepackage{ragged2e}
\hypersetup{
  colorlinks = true,
  allcolors=[rgb]{0,0.4,0.5},
}

%% Packages for Graphics & Figures
\usepackage{graphicx} %%For loading graphic files
\usepackage{float}

%% Math Packages
\usepackage{amsmath}
\usepackage{amsthm}
\usepackage{amsfonts}


\title{
  PhD project \\
  Linking forest management and species distribution models: \\
  a theoretical approach under climate change
}

\author[1,*]{Willian Vieira}
\affil[1]{Département de Biologie, Université de Sherbrooke, Sherbrooke, Québec, Canada}
\affil[*]{w.vieiraw@gmail.com}
\date{}

\begin{document}

\maketitle

\begin{abstract}

Abstract section

\end{abstract}

\thispagestyle{empty} %no page number in the first page

%logo
\vfill
\begin{figure}
\centering\includegraphics[width=16cm]{img/logo.pdf}
\end{figure}

\clearpage

\thispagestyle{empty} %no page number in the first page

\tableofcontents

\clearpage

\begin{displayquote}
\centering\textit{Ecology may provide many of the answers — but only if it is holistic enough to incorporate the human element as part and parcel of the ecosystem.} \\ \RaggedLeft{\citep[p. 231]{Pfister1993}}
\end{displayquote}

\section{What is going on?}

Climate change is an increasing trending topic both in non-scientific \citep{Capstick2015} and scientific environment (Figure \ref{fig:fig1}), transforming our world as a metamorphosis of practice and acting \citep{Beck2016}.
According to IPCC \citep{Cubasch2013}, humans activities are contributing to increase the concentration of greenhouse gases, which can lead to increase the mean temperature and the strength of extreme climate events.
This global change has an impact in different biological processes, from local species constraints \citep[e.g. low regeneration][]{Treyger2011}, shift in species' range \citep{Boisvert-Marsh2014,Monleon2015} and in community composition \citep{Dieleman2015} to range retractions and extinction \citep{Thomas2006}, impacting biodiversity at different scales \citep{Penuelas2013}.

\begin{wrapfigure}{r}{0.38\textwidth}
    \centering
    \includegraphics[width=0.38\textwidth]{img/fig1_em.pdf}
    \caption{Frequency of the keyword ``Climate change'' used in publication indexed on Google Scholar (1940 - 2013) and Web of Science (1994 - 2015)}
    \label{fig:fig1}
\end{wrapfigure}

Species distribution models (SDM; defined in section \ref{sdm}) is one of the most popular method to predict species' range shift under climate change, providing a wide range of applications, as in biodiversity conservation and management \citep{Guisan2005,Guisan2013}.
However, these models are generally phenomenological and  distributed at equilibrium with climate \citep[e.g.][]{Pigot2013}, being an issue when species observation does not reflect its niche \citep{Schurr2012}.
Hence, they do not consider important determinants of range limits as demography \citep{Louthan2015}, ecological constraints \citep{Wisz2013,Pigot2013} and species absences data \citep{Koshkina2017}, inducing non-accurately projection of the future spatial distribution of a species \citep{Tavecchia2016}.
Considering this ecological constraints, trees' migration rate following climate change will be slower than predicted \citep{Bertrand2011,Sittaro2017}, increasing the climatic debt \citep{Bertrand2016}.

%Extinction debt is following climate debt?
The climatic debt is a measure of the lag (or disequilibrium) of plant communities with climate change, integrated in an environmental context \citep{Bertrand2016}.
\citet{Essl2015} has listed twelve mechanisms that contribute to delayed biodiversity responses, among them, changes appears at ecosystem (loss and degradation), community (secessional, biotic interaction, species removal and invasion) and population (evolutionary and adaptive) levels.
Abiotic changes cause biotic changes that directly and indirectly promotes species' persistence and species' migration \citep{Bertrand2016}.
This mechanisms of persistence (measured by resistance) and migration (measured by recovery) leads to a climate debt and migration credit, respectively \citep{Bertrand2016} but also a concept of resilience \citep[see section \ref{res}]{Oliver2015}.
Therefore, this lag under climate change promotes extinction debt, being a challenging for biodiversity conservation \citep{Kuussaari2009} and productivity \citep{Lasch2002}.
Identify the mechanisms shaping delayed biotic response of systems to environment, its resilience as well as alternatives to mitigate ecological constraints, is crucial to access the vulnerability of biodiversity to climate change and improve forecasts and biodiversity management \citep{Essl2015,Oliver2015,Bertrand2016}.

\section{Preliminary objectives}
%I call this section as ``preliminary'' because I am still trying to figure out possible gaps I will be interested in working on.
%I also believe that classifying my plan as preliminary will help me identify throughout my thesis the ``best'' route to finish it in a pleasurable and creative way.

The primary objective of my thesis is to study if forest management can increase the speed of transition from temperate to boreal forests observed in the North Eastern America.
To achieve, I will use theoretical models parameterized from a forest inventory database, focusing on tracking uncertainty using Bayesian methods.
As an outcome, I will create a decision make tool to improve management strategies that take climate change into account.

My PhD is based in three questions so far: \\
(i) Which mechanisms are affecting the delayed biotic response to climate change? What is the origin, direction, intensity and interaction of these mechanisms? \\
(ii) How can forest management affect these mechanisms to speed up the response? \\
(iii) How can these mechanisms be used to inform applied management to enhance the resilience and productivity?

\section{Mitigating ecological constraints}

Ecological constraints act by different biotic and abiotic mechanisms at all scales, being difficult to track and therefore to mitigate.
To start, our work aims to identify how we can increase forest resilience, or more specific, how to decrease the recovery time of a system from a disturbance to a steady state (theories described in section \ref{ta}).
Here, I present some mechanisms that may be affecting the delayed response to climate change, as well as increase forest resilience, in which may become possible topics I will be testing during my thesis using SDM, acting both at local and large scale.

\subsection{Biotic Mechanisms}
Species with a high intrinsic rate of increase will recover more quickly from environmental perturbations (Mechanisms contributing to stability in ecosystem function depend on the environmental context. Ecol. Lett.)

\textbf{Species interaction}
Explain the role species interaction can play on its distribution range. \\
http://www.sciencedirect.com/science/article/pii/S0169534715002475
The perceived threat of climate change is often evaluated from species distribution models that are fitted to many species independently and then added together. This approach ignores the fact that species are jointly distributed and limit one another \citep{clark2014}.

Joint species distribution?

Interactions between land-use and climate also can underestimate species resilience in distribution models \citep{Goring2017}.

\subsection{Abiotic Mechanisms}

\subsection{Forest management}

Present some motivations, advantages and disadvantages in considering forest management.

It is clear that some factors will be more amenable to management (e.g., population-level genetic variability, landscape structure
[18,31]) than others (e.g., environmental sensitivity of individual species, presence of alternative stable states). In Olivier 2015

Four major strategies are available to mitigate carbon emissions through forestry activities: (i) to increase forested land area through reforestation (6), (ii) to increase the carbon density of existing forests at both stand and landscape scales, (iii) to expand the use of forest products that sustainably replace fossil-fuel CO2 emissions, and (iv) to reduce emissions from deforestation and degradation. in Managing forests for climate change mitigation (Science)

http://onlinelibrary.wiley.com/doi/10.1111/j.1752-4571.2010.00157.x/full \\
http://www.sciencedirect.com/science/article/pii/S0006320715301762

Interesting argument about who should peak do winners from \citet{Webster2017}. They say that \textit{Predict-and-prescribe management may erode diversity by focusing on ‘winners’}.

Forest management in a theoretical view was actually not very much explored and so I consider it a kind of gap we should better explore. Read \citet{Becknell2015} for an overview.

\subsection{Forest productivity}
Robert m'a envoyé quelques articles qui discutent des facteurs qui contrôlent la dominance des arbustes éricacées en forêt boréale et de leur possible impacte sur la productivité forestière.

\section{Theoretical approach}\label{ta}

Read the section \textit{recent developments in predicting changes in species distribution} from \citet{Ehrlen2015}

\subsection{Disturbance}
Pass through the main disturbance theories basing mainly in \citet{Pulsford2016}

\subsection{Resilience}\label{res}

A classical definition of resilience in ecology is the ability of ecosystems to absorb changes and still persist \citep{Holling1973}.
The concept was further developed in other context (e.g. social-ecological systems), and a more contemporary definition considers resilience as (i) the amount of disturbance the system can absorb, (ii) the degree the system is able to self-organize and (iii) the degree of learning capacity to adapt to disturbance \citep{Cumming2011}.
In other words, resilience is the joint of two concepts, the time to recovery to stability and accommodated external changes \citep{pimm1984,Folke2002} and the resistance of a particular ecological state to change \citep{Peterson1998}.
Resilience can be affected by different mechanisms from species to landscape levels, but biodiversity shows to be crucial to maintain long-term resilience of ecosystem services \citep{Oliver2015}.

It is also important to not confuse resilience with stability of a system. Resilience is the rate and extent of recovery of a system while stability is when the system maintains stable following small perturbations over time. Before introduce the method to calculate resilience, we must have an idea of what is the equilibrium (\hyperlink{box1}{Box 1}) and stability (\hyperlink{box2}{Box 2}) of a system.

\textbf{(iii)} But how to calculate it in a analytical way?  Now introduce the calculus of $\lambda$ by Jacobian matrix.

\begin{tcolorbox}
\hypertarget{box1}{Box 1}. Equilibrium
\begin{align}
E &= mc^2 & \text{Formula of the universe}
\end{align}
\end{tcolorbox}

\begin{tcolorbox}
\hypertarget{box2}{Box 2}. Stability
\begin{align}
E &= mc^2 & \text{Formula of the universe}
\end{align}
\end{tcolorbox}

\subsection{Transition period - Alternative Stable States?}

\subsection{Early warnings?}

\subsection{Range dynamics theory}

What theories can help us to describe species range under climate change?\\
How to integrate forest management in this theory?\\
-> Matapopulation dynamics theory <-

\section{Study case: the Quebec forest resource}

Explain here where I am going to work and also why I am choosing this area.

\section{Methods}

Here we see a briefly presentation of possible methods will be used in the thesis.

\subsection{Modeling}

Why and how modeling?\\
Morin et thuiller 2009:{What then are the best strategies for obtaining accurate predictions for changes in the distributions of deciduous temperate trees? At the scale of the geographic distribution of species, no experiments in situ can be reasonably carried out to predict possible range shifts (Woodward 1987). Modeling therefore appears the most feasible and efficient way to establish useful predictions (Lovejoy and Hannah 2005, Thuiller 2007), and several kinds of models have been developed during the previous decade for this purpose. As reviewed by Midgley et al. (2007), these models fall into two main classes: vegetation-type models (dynamic global vegetation models [DGVMs]) and species-specific models (niche-based and process-based).}

\subsubsection{Species Distribution Models}\label{sdm}

Nice resume about SDM in \citet{Moran-Ordonez2016}.

\subsubsection{Integral Projection Models}

the predictive powers of state and transition models are relatively low and their ability to deal with uncertainty is limited (Bashari, Smith and Bosch, 2008; Phillips, 2011).

\subsection{Bayesian approach}

\textit{I should be writing and not playing with \LaTeX}

\section{Thesis structure}

The first part of the thesis will be a general introduction in French where  I will probably use a part of this document and present the big picutre of my thesis.

The first chapter will try to answer the question \textit{Can forest management increase forest resilience to climate change?}. The paper will work with an analytical and sensitivity analysis in a metapopulation dynamics model to understand the impact of forest management on increasing forest resilience.

In the second chapter I am going to build a landscape model that will consider both forest management and species interaction.

The third chapter I am going to build another model but in a local scale. \textit{I have to find a good biological reason for that}.

The fourth chapter will then integrate both landscape and local model into one. Here I will also track the uncertainty of the model by bayesian approach.

\textbf{TODO}: \\
- Automate box labels \\
- Short reference style \\

\clearpage
\bibliography{/users/wvieira/Documents/mendeley_bibtex/Thesis.bib}

\end{document}
