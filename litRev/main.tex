% Literature review for the first meeting

\documentclass[a4paper,11pt,twoside]{article}
\usepackage[T1]{fontenc}
\usepackage[utf8]{inputenc}
\usepackage{color,dcolumn,graphicx,hyperref}
\usepackage{wrapfig}
\usepackage[top=2.5cm, bottom=2.5cm, left=2.5cm, right=2.5cm]{geometry}
\usepackage{setspace}
\onehalfspacing
\usepackage{scrextend}
\addtokomafont{labelinglabel}{\sffamily}
\usepackage[round]{natbib}
\bibliographystyle{abbrvnat}
\usepackage{tcolorbox}
\usepackage{filecontents}
\usepackage{csquotes}
\usepackage{epigraph}
\usepackage{ragged2e}
\usepackage{wrapfig}
\usepackage[font=small]{caption}
\hypersetup{
  colorlinks = true,
  allcolors=[rgb]{0,0.4,0.5},
}

%% Packages for Graphics & Figures
\usepackage{graphicx} %%For loading graphic files
\usepackage{float}

%% Math Packages
\usepackage{amsmath}
\usepackage{amsthm}
\usepackage{amsfonts}


\begin{document}

\title{Linking forest management and species distribution models: a theoretical approach under climate change}

\author{Willian Vieira}

\maketitle

\section{What is going on?}

\textbf{Climate change, obviously}

What is climate change?

Impacts of climate change (species range): Nice review in \cite{Price2013}

But also climate debit

============================================================

What is climate debit?

where and how it happens?

Example of how the speed of trees range is slow than climate change in \cite{Sittaro2017}. Another example is showed in \cite{Serra-Diaz2016} where little evidence that tree regeneration is shifting to higher latitudes and elevation (have to read carefully to better understand the results). \cite{Monleon2015} and \cite{Boisvert-Marsh2014} also shows evidence of tree species' range shift.

In the other hand, \cite{Malis2016} shows that observed range shifts among tree life stages are amore consistent with ontogenetic differences in the species' environmental requirements than with responses to recent climate change.

Impacts of climate debit (diversity, conservation, productivity?)

\section{What to do?}

Increase forest resilience! \textbf{(i)} But what is it exactly? (paper: Building evolutionary resilience for conserving biodiversity under climate change) Try and discuss this aspect in different perspectives. \textbf{(ii)} Why resilience?

Present some motivations, advantages and disadvantages in considering forest management.

Interesting argument about who should peak do winners from \cite{Webster2017}. They say that \textit{Predict-and-prescribe management may erode diversity by focusing on ‘winners’}.

Forest management in a theoretical view was actually not very much explored and so I consider it a kind of gap we should better explore. Read \cite{Becknell2015} for an overview.

\section{Study case: the Quebec forest resource}

Explain here where I am going to work and also why I am choosing this area.

\section{Theoretical approach}

Read the section \textit{recent developments in predicting changes in species distribution} from \cite{Ehrlen2015}

\subsection*{Resilience}
\textbf{(iii)} But how to calculate it in a analytical way? Before introduce the method to calculate resilience, we must have an idea of what is the equilibrium (\hyperlink{box1}{Box 1}) and stability (\hyperlink{box2}{Box 2}) of a model. Now introduce the calculus of $\lambda$ by Jacobian matrix.

\begin{tcolorbox}
\hypertarget{box1}{Box 1}. Equilibrium
\begin{align}
E &= mc^2 & \text{Formula of the universe}
\end{align}
\end{tcolorbox}

\begin{tcolorbox}
\hypertarget{box2}{Box 2}. Stability
\begin{align}
E &= mc^2 & \text{Formula of the universe}
\end{align}
\end{tcolorbox}

\subsection*{Range dynamics theory}

What theories can help us to describe species range under climate change?

How to integrate forest management in this theory?

\subsection*{Species Interaction: why is it important?}
Explain the role species interaction can play on its distribution range.

The perceived threat of climate change is often evaluated from species distribution models that are fitted to many species independently and then added together. This approach ignores the fact that species are jointly distributed and limit one another \citep{clark2014}.

Joint species distribution?

\section{How to do it?}

Here we see a briefly presentation of possible methods will be used in the thesis.

\subsection*{Modeling}

Why and how modeling?

\subsubsection*{Integral Projection Models}

\textit{I should be writing and not playing with \LaTeX}

\subsection*{Bayesian approach}

\textit{I should be writing and not playing with \LaTeX}

\section{Thesis structure}

The first part of the thesis will be a general introduction in French where  I will probably use a part of this document and present the big picutre of my thesis.

The first chapter will try to answer the question \textit{Can forest management increase forest resilience to climate change?}. The paper will work with an analytical and sensitivity analysis in a metapopulation dynamics model to understand the impact of forest management on increasing forest resilience.

In the second chapter I am going to build a landscape model that will consider both forest management and species interaction.

The third chapter I am going to build another model but in a local scale. \textit{I have to find a good biological reason for that}.

The fourth chapter will then integrate both landscape and local model into one. Here I will also track the uncertainty of the model by bayesian approach.

\bibliographystyle{plainnat}
\bibliography{/users/wvieira/Documents/mendeley_bibtex/Thesis.bib}

\end{document}
